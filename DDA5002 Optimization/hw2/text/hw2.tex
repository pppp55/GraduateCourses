\documentclass[12pt, a4paper, oneside]{article}
\usepackage[T1]{fontenc}
\usepackage{amsmath, amsthm, amssymb, bm, color, enumitem, graphicx, hyperref, mathrsfs, tikz, titling}
\usepackage[UTF8, scheme = plain]{ctex}
\usepackage{graphicx, minted, listings}
\title{\textbf{Homework 1}}
\setlength{\droptitle}{-10em}
\author{PENG Qiheng \\ Student ID\: 225040065}
\date{\today}
\linespread{1.5}
\newcounter{problemname}
\newenvironment{problem}{\stepcounter{problemname}\par\noindent{Problem \arabic{problemname}. }}{\par}
\newenvironment{solution}{\par\noindent{Solution. }}{\par}

\begin{document}

\maketitle

\begin{problem}
\begin{enumerate}[label = (\alph*)]
    \item The Linear program as follows:
    \begin{align}
        \notag \min_{x, z} \quad & c^T x + z \\
        \notag \text{s.t.} \quad & z \ge d^T x \\
        \notag & z \ge 0 \\
        \notag & z \ge 2d^T x - 4 \\
        \notag & Ax \ge b \\
        \notag & x \in \mathbb{R}^n, z \in \mathbb{R}
    \end{align}
\end{enumerate}
\begin{enumerate}[label = (\alph*)]
    \item The Linear program as follows:
    \begin{align}
        \notag \min_{x_2, z} \quad & 2x_2 + x_4 \\
        \notag \text{s.t.} \quad & x_1 - x_3 \le x_4 \\
        \notag & x_1 - x_3 \ge -x_4 \\
        \notag & x_1 + 2 \le x_5 \\
        \notag & x_1 + 2 \ge -x_5 \\
        \notag & x_2 \le x_6 \\
        \notag & x_2 \ge -x_6 \\
        \notag & x_5 + x_6 \le 5 \\
        \notag & x_3 \le 1 \\
        \notag & x_3 \ge -1 \\
        \notag & x_1, x_2, x_3, x_4, x_5, x_6 \in \mathbb{R}
    \end{align}
\end{enumerate}
\end{problem}

\newpage
\begin{problem}
    The standard form of the two linear program is as follows:
    \begin{align}
        \notag \min_x \quad & 5x^-_1 - 5x^+_1 -x_2 + 4x_3 \\
        \notag \text{s.t.} \quad & x^+_1 - x^-_1 + x_2 + x_3 - x_5 - x_6 = 19 \\
        \notag & 4x_2 - 8x_5 + x_7 = 45 \\
        \notag & x^+_1 - x^-_1 + 6x_2 - x_3 = 7 \\
        \notag & x^+_1, x^-_1, x_2, x_3, x_5, x_6, x_7 \ge 0
    \end{align}
    and
    \begin{align}
        \notag \min_x \quad & 2x_1 - 7x_2 + 6x_3 + 5x_4 \\
        \notag \text{s.t.} \quad & 2x_1 - 3x_2 - 5x_3 - 4x_4 + x_5 = 20 \\
        \notag & 7x_1 + 2x_2 + 6x_3 - 2x_4 = 35 \\
        \notag & 4x_1 + 5x_2 - 3x_3 - 2x_4 - x_6 = 15 \\
        \notag & x_1 + x_7 = 10 \\
        \notag & x_2 + x_8 = 8 \\
        \notag & x_3 - x_9 = 2 \\
        \notag & x_1, x_2, x_3, x_4, x_5, x_6, x_7, x_8, x_9 \ge 0
    \end{align}
\end{problem}

\newpage
\begin{problem}
    The graphic is as follows: \\
    \begin{tikzpicture}[scale=0.8]
		\draw [->] (-4,0)--(11,0) node[below right] {$x_1$};
		\draw [->] (0,-2)--(0,6) node[above left] {$x_2$};
		\node[below left] at (0,0) {0};
		\node[below] at (-2.5,0) {$-2.5$};
		\node[below] at (4,0) {$4$};
		\node[below] at (9,0) {$9$};
		\node[left] at (0,2.5) {$2.5$};
		\node[left] at (0,2) {$2$};
		\node[left] at (0,4.5) {$4.5$};
        \draw[black] (-4,-1.5) -- (3.5,6);
        \draw[black] (-3,6) -- (11,-1);
        \draw[black] (-4,3) -- (11,3);
        \draw[black] (4,-2) -- (4,6);
        \fill[green, opacity=0.2] (0,0) -- (0,2.5) -- (0.5,3) -- (3,3) -- (4,2.5) -- (4,0);
        \fill [red] (0,0) circle (3pt);
        \fill [red] (0,2.5) circle (3pt);
        \fill [red] (0.5,3) circle (3pt);
        \fill [red] (3,3) circle (3pt);
        \fill [red] (4,2.5) circle (3pt);
		\node[right, red] at (4,2.5) {optimal};
        \fill [red] (4,0) circle (3pt);
    \end{tikzpicture}
    \newline All the vertices of the feasible region are: \\
    (0, 0), (0, 2.5), (0.5, 3), (3, 3), (4, 2.5), (4, 0). \\
    The optimal solution is $(4, 2.5, 6.5)$, while the active constraints are \\
    $x_1 + x_2 - x_3 = 0$, $-x_1 + 2x_2 \le 2.5$ and $x_1 + 2x_2 \le 9$.
\end{problem}

\newpage
\begin{problem}
    The standard form of the linear program is as follows:
    \begin{align}
        \notag \min_x \quad & -x_1 - x_2 \\
        \notag \text{s.t.} \quad & x_1 + 3x_2 - x_3 = 15 \\
        \notag & 2x_1 + x_2 - x_4 = 10 \\
        \notag & x_1 + 2x_2 + x_5 = 40 \\
        \notag & 3x_1 + x_2 + x_6 = 60 \\
        \notag & x_1, x_2, x_3, x_4, x_5, x_6 \ge 0
    \end{align}
    Simplex method process by the linear algebra derivation way is as follows: \\
    Iteration 1:
    We choose basis $\{1, 2, 5, 6\}$
    \begin{gather}
        \notag A_B = \begin{bmatrix}
            1 & 3 & 0 & 0 \\
            2 & 1 & 0 & 0 \\
            1 & 2 & 1 & 0 \\
            3 & 1 & 0 & 1
        \end{bmatrix}, \quad
        A_B^{-1} = \begin{bmatrix}
            -\frac{1}{5} & \frac{3}{5} & 0 & 0 \\
            \frac{2}{5} & -\frac{1}{5} & 0 & 0 \\
            -\frac{3}{5} & -\frac{1}{5} & 1 & 0 \\
            \frac{1}{5} & -\frac{8}{5} & 0 & 1
        \end{bmatrix} \\
        \notag x_B = A_B^{-1} b = \begin{bmatrix}
            3 \\
            4 \\
            29 \\
            47
        \end{bmatrix} \\
        \notag \bar c_3 = c_3 - c_B^T A_B^{-1} A_3 = -\frac{1}{5} \\
        \notag \bar c_4 = c_4 - c_B^T A_B^{-1} A_4 = -\frac{2}{5}
    \end{gather}
    So we choose $x_4$ to enter the basis.
    \begin{gather}
        \notag d_B = -A_B^{-1} A_4 =
        \begin{bmatrix}
            \frac{3}{5} \\
            -\frac{1}{5} \\
            -\frac{1}{5} \\
            -\frac{8}{5}
        \end{bmatrix} \\
        \notag \theta^* = \min_{i \in B, d_i < 0} \frac{-x_i}{d_i} = \frac{-x_2}{d_2} = 20
    \end{gather}
    Therefore, $x_2$ exits the basis. \\
    Iteration 2:
    We choose basis $\{1, 4, 5, 6\}$
    \begin{gather}
        \notag A_B = \begin{bmatrix}
            1 & 0 & 0 & 0 \\
            2 & -1 & 0 & 0 \\
            1 & 0 & 1 & 0 \\
            3 & 0 & 0 & 1
        \end{bmatrix}, \quad
        A_B^{-1} = \begin{bmatrix}
            1 & 0 & 0 & 0 \\
            2 & -1 & 0 & 0 \\
            -1 & 0 & 1 & 0 \\
            -3 & 0 & 0 & 1
        \end{bmatrix} \\
        \notag x_B = A_B^{-1} b = \begin{bmatrix}
            15 \\
            20 \\
            25 \\
            15
        \end{bmatrix} \\
        \notag \bar c_2 = c_2 - c_B^T A_B^{-1} A_2 = 2 \\
        \notag \bar c_3 = c_3 - c_B^T A_B^{-1} A_3 = -1
    \end{gather}
    So we choose $x_3$ to enter the basis.
    \begin{gather}
        \notag d_B = -A_B^{-1} A_3 =
        \begin{bmatrix}
            1 \\
            2 \\
            -1 \\
            -3
        \end{bmatrix} \\
        \notag \theta^* = \min_{i \in B, d_i < 0} \frac{-x_i}{d_i} = \frac{-x_6}{d_6} = 5
    \end{gather}
    Therefore, $x_6$ exits the basis. \\
    Iteration 4:
    We choose basis $\{1, 2, 3, 4\}$
    \begin{gather}
        \notag A_B = \begin{bmatrix}
            1 & 3 & -1 & 0 \\
            2 & 1 & 0 & -1 \\
            1 & 2 & 0 & 0 \\
            3 & 1 & 0 & 0
        \end{bmatrix}, \quad
        A_B^{-1} = \begin{bmatrix}
            0 & 0 & -\frac{1}{5} & \frac{2}{5} \\
            0 & 0 & \frac{3}{5} & -\frac{1}{5} \\
            -1 & 0 & \frac{8}{5} & -\frac{1}{5} \\
            0 & -1 & \frac{1}{5} & \frac{3}{5}
        \end{bmatrix} \\
        \notag x_B = A_B^{-1} b = \begin{bmatrix}
            16 \\
            12 \\
            37 \\
            34
        \end{bmatrix} \\
        \notag \bar c_5 = c_5 - c_B^T A_B^{-1} A_5 = \frac{2}{5} \\
        \notag \bar c_6 = c_6 - c_B^T A_B^{-1} A_6 = \frac{1}{5}
    \end{gather}
    So the current BFS is optimal. Optimal solution is $(16, 12, 37, 34, 0, 0)$ and optimal value is $-28$. \\
    \begin{tikzpicture}[scale=0.22]
		\draw [->] (-5,0)--(50,0) node[below right] {$x_1$};
		\draw [->] (0,-5)--(0,70) node[above left] {$x_2$};
		\node[below left] at (0,0) {0};
        \draw[black] (-5,20/3) -- (50,-35/3);
		\node[left] at (0,5) {$5$};
		\node[below] at (15,0) {$15$};
        \draw[black] (-5,20) -- (7.5,-5);
		\node[left] at (0,10) {$10$};
		\node[below] at (5,0) {$5$};
        \draw[black] (-5,45/2) -- (50,-5);
		\node[left] at (0,20) {$20$};
		\node[below] at (40,0) {$40$};
        \draw[black] (-10/3,70) -- (65/3,-5);
		\node[left] at (0,60) {$60$};
		\node[below] at (20,0) {$20$};
        \fill[green, opacity=0.2] (3,4) -- (0,10) -- (0,20) -- (16,12) -- (20,0) -- (15,0);
        \fill [red] (3,4) circle (15pt);
        \draw [red, ->] (3,4)--(9,2);
        \fill [red] (15,0) circle (15pt);
        \draw [red, ->] (15,0)--(17.5,0);
        \fill [red] (20,0) circle (15pt);
        \draw [red, ->] (20,0)--(18,6);
        \fill [red] (16,12) circle (15pt);
		\node[right, red] at (16,12) {optimal};
    \end{tikzpicture}
\end{problem}

\newpage
\begin{problem}
\begin{enumerate}[label = (\alph*)]
    \item The auxiliary LP of Phase I is as follows:
    \begin{align}
        \notag \min_y \quad & y_1 + y_2 \\
        \notag \text{s.t.} \quad & x_1 - x_2 - 2x_3 - s_1 + y_1 = 2 \\
        \notag & x_2 - x_3 + 2x_4 + s_2 = 4 \\
        \notag & 2x_1 + 3x_3 - x_4 + y_2 = 2 \\
        \notag & x_1, x_2, x_3, x_4, s_1, s_2, y_1, y_2 \ge 0
    \end{align}
    \item With the basis of $\{x_1, x_4, y_2\}$:
    \begin{gather}
        \notag A_B = \begin{bmatrix}
            1 & 0 & 0 \\
            0 & 2 & 0 \\
            2 & -1 & 1
        \end{bmatrix}, \quad
        A_B^{-1} = \begin{bmatrix}
            1 & 0 & 0 \\
            0 & \frac{1}{2} & 0 \\
            -2 & \frac{1}{2} & 1
        \end{bmatrix} \\
        \notag \begin{bmatrix}
            x_1 \\
            x_4 \\
            y_2
        \end{bmatrix} = A_B^{-1} b =
        \begin{bmatrix}
            2 \\
            2 \\
            0
        \end{bmatrix}
    \end{gather}
    While $y_1 = y_2 = 0$, this solution is optimal for the auxiliary LP of Phase I.
    \item Follow (b), we have:
    \begin{gather}
        \notag \bar c_{x_2} = c_{x_2} - c_B^T A_B^{-1} A_{x_2} = -\frac{5}{2} \\
        \notag \bar c_{x_3} = c_{x_3} - c_B^T A_B^{-1} A_{x_3} = \frac{3}{2} \\
        \notag \bar c_{s_1} = c_{s_1} - c_B^T A_B^{-1} A_{s_1} = -2 \\
        \notag \bar c_{s_2} = c_{s_2} - c_B^T A_B^{-1} A_{s_2} = -\frac{1}{2} \\
        \notag \bar c_{y_1} = c_{y_1} - c_B^T A_B^{-1} A_{y_1} = 3
    \end{gather}
    So we choose $x_2$ to enter the basis.
    \begin{gather}
        \notag d_B = -A_B^{-1} A_2 =
        \begin{bmatrix}
            1 \\
            -\frac{1}{2} \\
            -\frac{5}{2}
        \end{bmatrix} \\
        \notag \theta^* = \min_{i \in B, d_i < 0} \frac{-x_i}{d_i} = \frac{-y_2}{d_{y_2}} = 5
    \end{gather}
    Therefore, $y_2$ exits the basis, and the new basis in the next iteration is $\{x_1, x_2, x_4\}$.
\end{enumerate}
\end{problem}

\begin{problem}
\begin{enumerate}[label = (\alph*)]
    \item False. Assume a LP as follows:
    \begin{align}
        \notag \min_x \quad & x_1 + x_2 \\
        \notag \text{s.t.} \quad & x_1 + x_2 - x_3 \ge 1 \\
        \notag & x_1, x_2, x_3 \ge 0
    \end{align}
    The optimal solution is $\{x_1, x_2, x_3 | x_1 + x_2 = 1\}$, which is unbounded.
    \item False. As LP in (a), one of the optimal solution is $(0, 1, 1)$, while $m = 1$. However, it has more than $m$ variables being positive.
    \item True.
    \item False. As LP in (a), one of the optimal solution $(0, 0.5, 0.5)$ is not a basic feasible solution.
\end{enumerate}
\end{problem}

\end{document}